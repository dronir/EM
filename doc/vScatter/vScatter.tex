\documentclass[a4paper]{article}
\usepackage[T1]{fontenc}
\usepackage[utf8]{inputenc}
\usepackage{lmodern}
\usepackage{amssymb, amsmath}
\usepackage{verbatim}
\usepackage{listings}
\usepackage{textcomp}
\usepackage[english]{babel}
\usepackage[pdftex]{color, graphicx}

\providecommand{\abs}[1]{\lvert#1\rvert}

\newcommand{\code}[1]{\lstinline{#1}}
\lstset{basicstyle=\ttfamily\small, numbers=left, tabsize=4}

\newcommand{\this}[0]{\texttt{vScatter}}
\newcommand{\prog}[1]{\texttt{#1}}

\begin{document}
	
\title{The \this~manual}
\author{Olli Wilkman}

\maketitle

\tableofcontents

\section{Overview}

\this~is a program for calculating shadowing effects in visible light scattering from particulate media. It works in a variety of different modes.

\subsection{Shadowing}

When calculating shadowing, multiple scattering or scattering laws are not considered. The simulation only checks whether a tested point is lit or not, and gives the ratio of lit points to all checked points, which approximates the shadowing function $S(\mu_0, \mu, \phi)$. This means that every lit point contributes an equal amount and $I(\mu_0, \mu, \phi)=1$ always.

\subsection{Lambertian scattering}

In Lambertian scattering

\[
	I(\mu_0, \mu, \phi) = \frac{\omega}{\pi} \mu_0,
\]

where $\omega$ is the single-scattering albedo.



\subsection{Lommel-Seeliger scattering}\label{LommelSeeliger}

The Lommel-Seeliger scattering law as implemented in this program is

\[
	I(\mu_0, \mu, \phi) = \frac{\omega}{4 \pi} \frac{\mu_0}{\mu_0 + \mu} P_V(\alpha),
\]

where $\omega$ is the single-scattering albedo and $P_V(\alpha)$ is a phase function. 

The phase function can be either a constant $P_V(\alpha) = 1$, a single Henyey-Greensein function or a a sum of two Henyey-Greenstein functions. The Henyey-Greenstein function is defined as

\[
	\text{HG}(\alpha ; g) = \frac{1 - g^2}{(1 + g^2 - 2 g \sin \alpha)},
\]

where $g$ is the \emph{asymmetry parameter}.

The double Henyey-Greenstein function is defined as 

\[
	\text{HG}_2(\alpha ; w, g_1, g_2) = w \frac{1 - g_1^2}{(1 + g_1^2 - 2 g_1 \sin \alpha)}
	+ (1-w) \frac{1 - g_2^2}{(1 + g_2^2 - 2 g_2 \sin \alpha)}
\]

where $g_1$ and $g_2$ are the asymmetry parameters of the two components and $w$ defines their relative weights.


\subsection{Radiative Transfer}

I'm not sure what this is, but it's defined as 

\[
	I(\mu_0, \mu, \phi) = \frac{\omega}{4 \pi} \frac{\mu_0}{\mu_0 + \mu} P_V(\alpha) H(\mu_0, \omega) H(\mu, \omega),
\]

where 
\begin{align}
H(\mu, \omega) =& \left( 1 - \omega \mu \left(r_0 + \frac{1}{2} \left(1 - 2 r_0 \mu \right) \log \left(1 + \frac{1}{x} \right)\right) \right)^{-1} \\
r_0 =& \frac{1-\sqrt{1- \omega}}{1+\sqrt{1- \omega}}
\end{align}

and $P_V$ works exactly like in the Lommel-Seeliger case.

\subsection{Monte Carlo}

This is something else altogether.



\section{Input file}\label{input}

An example \this~input file can be seen in Listing \ref{inputlst}.

\textbf{Important:} It is not clear from Listing \ref{inputlst}, because the \LaTeX~package used to produce it removes trailing whitespace, but there \emph{must} be a newline or a space after the slash which ends the input file on the last line. You will get an error about an unexpected end-of-file if the EOF is right after the slash.

The input file must start with \code{&params} and end in a slash. In between are parameter value assignments separated by commas. Most of the parameter names are quite self-describing. Most of them have default values, which are used if the parameters are not specified in the file.

It is not necessary to define all parameters in the input file. In that case a default value is used. This is useful because many of the possible parameters do not seem to be used by the program.

\begin{description}
	\item[gridResHorizontal]
	\item[gridResVertical]
	\item[hsCarPlotFilename] is not used.
	\item[hsSphPlotFilename] is not used.
	\item[hsFilename] is the name of the output (``hemisphere'') file.
	\item[mediumFilename] is the path to the medium file used by the simulation.
	\item[sDistribution] is not used at the moment.
	\item[rho] is not used at the moment.
	\item[rhoAllowedError] is not used at the moment.
	\item[sampleSeed] does not seem to be used by the program.
	\item[resThetaE] is the $\theta_e$ resolution, that is, the number of different $\theta_e$ values used. The output hemisphere is binned binned in this many equal-width bins along the $\theta_e$ direction.
	\item[nThetaIn] is the $\theta_i$ resolution.
	\item[thetaIn] is a string giving a list of incidence angles. There must be as many numbers as given by \code{nThetaIn}.
	\item[nOrders] is the number of scattering orders that the ray is followed to.
	\item[nSamplesPerOrder] is a string giving the numbers of raytracing samples for each scattering order. If \code{nOrders = 3}, then this parameter must contain three numbers, separated by a space, for example \code{nSamplesPerOrder =} \code{"50 20 10"}. The default value is \code{"1"}.
	\item[mu] is the extinction coefficient $\mu$.
	\item[w] is the single scattering albedo $\omega$.

	\item[brdfType] gives the type of BRDF function. The different options are \code{shadowing}, \code{Lambert}, \code{LommelSeeliger}, \code{RadTransfer} and \code{MonteCarlo}. %See Section TODO for details.
	\item[brdfPhaseFunction] can be supplied if \code{brdfType} is \code{LommelSeeliger} or \code{RadTransfer}. The options are \code{constant}, \code{HG1} and \code{HG2}. %See Section TODO for details.
	\item[pfParams] is a string giving the phase function parameters, separated by spaces. The interpretation of the parameters depends on which phase function is chosen. If \code{brdfPhaseFunction} is \code{HG1}, the first parameter is used as the asymmetry parameter of the HG function. If \code{brdfPhaseFunction} is \code{HG2}, three parameters are required and the order is $w$, $g_1$, $g_2$ (see Section \ref{LommelSeeliger} for definitions). If \code{brdfPhaseFunction} is \code{constant}, no parameters are needed.
	\item[nPfParams] gives the number of phase function parameters. It must be set equal to the number of parameters given in the \code{pfParams} string, or the code is not able to read the parameters correctly
	
	\item[rf\_applyFields] is a Boolean value telling whether or not to apply a random field to the medium surface to simulate surface roughness. In most cases this is what you want to do. A random surface is generated and the medium is clipped by removing all spheres above that surface. The value should be either \code{.true.} or \code{.false.}. The default value is \code{.false.}.
	\item[rf\_nFieldsPerMed] gives the number of random field realizations to use for the medium. It is a good idea to use several for better statistics. The default value is \code{1}.
	\item[rf\_spectrumType] gives the type of surface statistic used to generate the random field. It can be either \code{Gaussian} for Gaussian statistics, \code{fBm} for fractal statistics or \code{constant} for clipping by a plane surface. The default value is \code{"Gaussian"}.
	\item[rf\_P] is a parameter of the random field and its meaning is different for each type. For Gaussian statistics it is the correlation length $l$. For fBm statistics it is the Hurst exponent $H$. In both of these cases it controls the horizontal scale of the surface roughness. For a planar surface it is the fraction of the medium height at which it is clipped. The default value is \code{0.5}. % TODO: add references
	\item[rf\_std] is the standard deviation parameter $\sigma/L$ which controls the amplitude of the random field. It gives the standard deviation of the heights relative to the horizontal width of the medium. The default value is \code{1.0}.
	\item[nThreads] would be used for distributing. How this works, I don't currently know. % TODO: figure out how the multitasking works
	\item[mediumMapRes]
	\item[mediumDensitymapRes]
\end{description}

\begin{lstlisting}[float, frame=trbl, caption="Example input file", label=inputlst]
&params
  resThetaE = 45,
  gridResHorizontal = 200,
  gridResVertical = 10,

  hsCarPlotFilename = "vScatter_car.eps",
  hsSphPlotFilename = "vScatter_sph.eps",
  hsFilename = "hemisphere.nc",

  mediumFilename = "medium.bc",
  sDistribution = "constant",

  rho = 0.50_fd,
  rhoAllowedError = 0.05_fd,

  sampleSeed = 0,

  nThetaIn = 1,
  thetaIn = "45.0",

  nOrders = 1,
  nSamplesPerOrder = "1",

  mu = 1.0,
  w = 0.1,

  nPfParams = 1,
  pfParams = "0.0",

  brdfType = "shadowing",
  brdfPhaseFunction = "constant",

  rf_applyFields = .false.,
  rf_nFieldsPerMed = 1,
  rf_spectrumType = "Gaussian",
  rf_P = 0.5_fd,
  rf_std = 1.0_fd,

  nThreads = 1,
  mediumMapRes = 256,
  mediumDensitymapRes = 300,
/

\end{lstlisting}

\section{Medium file}\label{medium}

The medium file is any output file from the \code{medgen} program. See the \code{medgen} documentation for more information. 

When doing these computations, make sure that the medium file is thick enough that no rays escape through it.

Also note that a medium file can contain several media. The simulation will be run for each medium, and the results will be \emph{combined} in the output.
% TODO: figure out multiple media behaviour for vScatter
% TODO: add suggestion for medgen parameter values


\section{Output file}\label{output}




\end{document}